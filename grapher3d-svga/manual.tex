\magnification = \magstep 2
\font\bigrm = cmr10 scaled \magstep 3
\font\medium = cmr10 scaled \magstep 2
\font\small  = cmr10 scaled \magstep 0
\nopagenumbers
\vglue 1 in

\centerline{\bf \bigrm Grapher-3D}
\vskip 25 pt
\centerline{\bf \medium User's Manual}

\vskip 250 pt

\hrule
\vskip 10 pt
\small Michael Paul Mesnier

\small Northeast Missouri State University

\small Kirksville, Mo. 63501

\vfill
\eject
\vglue .5 in
\centerline{\bf \bigrm Table of Contents}                               
\vskip 30 pt

             \hskip 85 pt I.~~~Project's Purpose.
                          
\vskip 20 pt \hskip 85 pt II.~~Functionality.

\vskip 20 pt \hskip 85 pt III.~Use.

\vskip 20 pt \hskip 85 pt IV.~Program Bugs. (Boo...Hiss)

\vfill
\eject
\vglue .5 in
\centerline{\bf \bigrm Project's Purpose}
\vskip 24 pt

Grapher-3D provides users with a visual aid to three dimensional 
functions.  Unlike two dimensional functions, the complexity of a three 
dimensional function prevents an easy sketching by hand.  
Further, Grapher-3D allows users to quickly change their 
position and ``view'' the function from any point in three-space.

\vfill
\eject
\vglue .5 in
\centerline{\bf \bigrm Project's Functionality}                           
\vskip 24 pt

Grapher-3D has made enormous strides since the inception of 
my ``Grapher'' projects.  The original Grapher-2D, coded entirely 
in Apple Pascal, only accomodated a single, hard-coded function,  
and viewing new functions necessarily required recompilation.  
Grapher-3D, however, allows numerous three-dimensional functions 
to be plotted without ever leaving the application.
      
\vfill
\eject
\vglue .25 in
\centerline{\bf \bigrm Use of Grapher-3D}
\vskip 24 pt
\centerline{\bf I. Plotting a surface.}
\vskip 12 pt

By default, Grapher-3D begins in graphics mode with an 
interesting surface preloaded for your enjoyment.  You can 
either experiment with this surface or escape to the main 
menu and create your own.  Find listed below a description 
of the hot keys available while viewing a surface.  Note that
these keys are also described in `Instructions' from the main
menu.

\vskip 12 pt
\settabs 16 \columns

\+x &: Rotates surface about the $x$ axis.\cr

\+y &: Rotates surface about the $y$ axis.\cr

\+z &: Rotates surface about the $z$ axis.\cr

\vskip 6 pt

\noindent
Note: The resolution of a surface is the number of points 
that constitute each line of the graph.  Smooth ``animation'' 
can be achieved with lower resolutions, where finer looking 
surfaces require higher resolutions.

\vskip 6 pt

\+c &: Changes the surfaces color. \cr

\noindent
(An option I included for all 
the users who find it necessary to ask ``Can it change colors?'')

\vskip 6 pt

\+1 &: Stretches $z$ axis.\cr

\+2 &: Shrinks $z$ axis.\cr

\+3 &: Turns range checking on.\cr

\+4 &: Turns range checking off. (default)\cr

\+5 &: Decreases the resolution of the surface.\cr

\+6 &: Increases the resolution of the surface.\cr

\vskip 6 pt

\noindent
Note: Range checking prevents surfaces from ``wrapping'' around 
the screen.

\vskip 6 pt

\+ $+$ &: Enlarges surface. \cr

\+ $-$ &: Reduces surface. \cr

\+ m &: Exit to main menu.\cr

\+ q &: Quit Grapher-3D. \cr

\vfill
\eject
\centerline{\bf II. Creating a New Funtion}
\vskip 12 pt

Creating functions in Grapher-3D requires a familiarity 
with postfix notation (later releases will accept infix 
expressions).  With little practice, postfix expressions 
will become second nature.  Included below are the defined 
functions in Grapher-3D along with a few examples.

\vskip 12 pt
     
\+ x   &: denotes the variable $x$ in the function $f(x,y)$.\cr
      
\+ y   &: denotes the variable $y$ in the function $f(x,y)$.\cr
       
\+ +   &: addition         &&&&& (e.g. $x+y$ = xy+).\cr

\+ $-$ &: subtraction      &&&&& (e.g. $x-y$ = xy-).\cr
 
\+ *   &: multiplication   &&&&& (e.g. $x*(x+y)$ = xy+x*).\cr
 
\+ /   &: division         &&&&& (e.g. $x/(x*y)$ = xy*x/).\cr

\+ \^{}&: exponents        &&&&& (e.g. $x^2$ = x2\^{}).\cr

\+ s   &: sine             &&&&& (e.g. $\sin(x*y)$ = xy*s).\cr

\+ c   &: cosine           &&&&& (e.g. $\cos(x*y-1)$ = xy*1-c).\cr

\+ e   &: exponent base e  &&&&& (e.g. $\exp(x/y)$ = xy/e).\cr

\+ l   &: natural log      &&&&& (e.g. $\ln(x-y)/(x+1)$ = xy-lx1+/).\cr

\+ n   &: negation         &&&&& (e.g. $-(x+y)$ = xy+n).\cr

\+ a   &: absolute value   &&&&& (e.g. $abs(x)$ = xa).\cr

\vskip 6 pt

\noindent
Note about constants: Only integer constants are allowed.  

\vskip 12 pt
\centerline{\bf III.  Domain}
\vskip 6 pt

After selecting `Domain' from the main menu, you will be 
prompted to enter the minimum and maximum x values along with 
the minimum and maximum y values.  Both real and integer values 
are allowed.

\vskip 12 pt
\centerline{\bf IV.  Instructions}
\vskip 6 pt

Choosing `Instructions' from the main menu will provide the user 
with a screenful of helpful information describing the available 
hot keys while in graphics mode.

\vfill
\eject
\vglue .5 in
\centerline {\bf \bigrm Program Bugs}
\vskip 12 pt

Be careful when using exponential functions, they tend to grow 
extremely fast and produce overflow and/or underflow errors. 
Later releases will have run-time checks on function arguments. 
Also, when using higher resolutions, the super-vga library has 
been known to crash with a segmentation violation (ouch!).

\bye
